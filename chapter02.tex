\chapter{Conceitos Iniciais}

\section{Interface Gr\'afica (UI)}
Interface gráfica, comumente abreviada para UI (do inglês User Interface) é a 
parte da aplicação que o usuário vê e com a qual ele interage. \cite{Chong2004}
A interface inclui as telas, janelas, controles, menus e etc. \cite{Chong2004}

De acordo com um estudo publicado pelo IEEE \cite{Khalid2015} sobre os motivos pelos 
quais os usuários mais reclamam de aplicações móveis, a interface gráfica é um dos 
casos mais frequentes. O que demonstra a necessidade de empenho no desenvolvimento
de interfaces agradáveis e intuitivas.

Não é de se estranhar que as empresas desenvolvedoras dos sistemas operacionais
cujas plataformas móveis são as mais competitivas no mercado ofereçam maneiras de
padronizar as interfaces dos aplicativos feitos para suas plataformas.

De fato, a Google e a Apple oferecem dentro de seus SDKs ferramentas de padronização
de interfaces. No caso da Google, esse conjunto de ferramentas
é chamado de Material Design e, além de ser oferecido para o Android, também
pode ser utilizada nas plataformas Web, Flutter e, inclusive, iOS.
Já a Apple, oferece seu kit gráfico, o UIKit, apenas para sua própria plataforma.

\section{Experiência do Usu\'ario (UX)}