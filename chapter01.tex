\chapter{Introdução}

O Restaurante Universitário da UFRJ tem como objetivo oferecer alimentação de qualidade, equilibrada, e 
acessível de forma a favorecer a permanência dos estudantes no espaço universitário, permitindo-lhes 
dedicação integral aos estudos, sendo importante meio de combate à evasão escolar.

Entretanto, apesar de ser benéfico à comunidade universitária, o RU enfrenta alguns problemas, sendo o 
principal as filas de espera. Pensando nisso, a Decania do Centro de Tecnologia resolveu criar um sistema
 de agendamento online cujo objetivo é alocar os horários de entrada no RU.

Esse projeto funciona desde 2016 na unidade do RU do CT. Os agendamentos são feitos através do website
www.ru.ct.ufrj.br. Entretanto, para melhorar ainda mais a experiência dos usuários, a Decania do CT 
resolveu desenvolver aplicativos para smartphone tendo em vista a crescente popularização desta 
plataforma, notoriamente para Android e para iOS.

Desde então, deu-se inicio ao projeto de desenvolvimento de aplicativos nativos. Esta 
monografia mostrará um comparativo entre essas duas ferramentas e utilizará como ilustração
os aplicativos desenvolvidos para o Restaurante Universitário da UFRJ.