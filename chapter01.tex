\chapter{Introdução}
\section{Tema}

O tema deste trabalho é o desenvolvimento de aplicativos móveis. Neste sentido, os problemas a serem 
resolvidos são a viabilidade em criar aplicações para plataformas móveis, 
bem como,  a capacidade destas aplicações de ampliar a acessibilidade aos serviços 
oferecidos e de melhorar a experiência de uso dos usuários.

\section{Delimitação}

O objetivo do projeto é atender aos usuários e aos funcionários do Restaurante 
Universitário do Centro de Tecnologia e de Letras da UFRJ. O escopo de usuários é variável sendo 
composto pelos alunos, professores e servidores da instituição, bem como por eventuais visitantes que 
vão à universidade para congressos, semanas esportivas, dentre outros. O projeto também tem 
capacidade de atender às outras unidades do RU da UFRJ, bem como ser adaptado para outras 
universidades.

\section{Justificativa}

O Restaurante Universitário da UFRJ tem como objetivo oferecer alimentação de qualidade, equilibrada, e 
acessível de forma a favorecer a permanência dos estudantes no espaço universitário, permitindo-lhes 
dedicação integral aos estudos, sendo importante meio de combate à evasão escolar.

Entretanto, apesar de ser benéfico à comunidade universitária, o RU enfrenta alguns problemas, sendo o 
principal as filas de espera. Pensando nisso, a Decania do Centro de Tecnologia resolveu criar um sistema
 de agendamento online cujo objetivo é alocar os horários de entrada no RU e, desta forma, tornar as 
 filas, que antes eram físicas e que geravam desgaste aos alunos, em filas virtuais.

Esse projeto funciona desde 2016 na unidade do RU do CT. Os agendamentos são feitos através do website
www.ru.ct.ufrj.br. Entretanto, para melhorar ainda mais a experiência dos usuários, bem 
como para ampliar a acessibilidade aos serviços prestados pelo RU, a Decania do CT 
resolveu desenvolver aplicativos nativos para smartphone tendo em vista a crescente popularização desta 
modalidade, notoriamente para as plataformas Android e iOS.

\section{Objetivo}
O objetivo geral é, portanto, desenvolver aplicações móveis nativas em Android (Java) e em iOS (Swift) 
capazes de fornecer os serviços virtuais do Restaurante Universitário, sendo o principal deles
o agendamento de horários para entrar no restaurante. Desta forma, tem-se como objetivos específicos 
principais: 
\begin{itemize}
\setlength\itemsep{0em}
\item Permitir a criação, edição e exclusão de filas virtuais para o acesso ao RU por parte de seus funcionários; 
\item Permitir o agendamento em filas virtuais de acesso ao RU, bem como sua exclusão ou verificação de 
status por parte dos alunos;
\item Visualização das filas virtuais vigêntes por parte de todos os usuários do RU.
\end{itemize}

\section{Metodologia}
Este trabalho utilizou os kits de desenvolvimento de software (SDK, do inglês Software Development 
Kit) oficiais oferecidos pela Google e pela Apple para suas plataformas Android e iOS, respectivamente.

Atendendo às boas práticas de programação e aos conceitos de orientação a objetos,
foram desenvolvidos três aplicativos móveis. São eles:
\begin{itemize}
\setlength\itemsep{0em}
\item Aplicativo Android para clientes do RU;
\item Aplicativo iOS para clientes do RU;
\item Aplicativo Android para funcionários do RU.
\end{itemize}

As etapas da criação de cada aplicação consistem em:
\begin{itemize}
\setlength\itemsep{0em}
\item Criação de uma interface para cada funcionalidade;
\item Criação da lógica de interação com o usuário para alternar entre as 
interfaces;
\item Criação de métodos para a comunicação com o Back-End;
\item Atualização dos dados exibidos na interface ou alternância de interface
 de acordo com a resposta do Back-End.
\end{itemize}

 A proposta deste trabalho é mostrar como cada etapa do projeto foi 
 desenvolvida para os aplicativos para os clientes do RU com respaldo 
 na literatura de desenvolvimento de aplicativos para Android e para iOS.
 O aplicativo para os funcionários foi desenvolvido apenas para a 
 plataforma Android.
 
 Além disso, são expostas as possíveis tecnologias a serem usadas para a 
 comunicação com o Back-End, como por exemplo, os estilos de arquitetura de 
 serviços REST versus SOAP em APIs. O desenvolvimento da 
 API de comunicação e do Back-End, entretanto, não fazem parte do escopo deste projeto, 
 pois os mesmos já estavam em desenvolvimento por outro aluno desde o início do projeto.

O êxito deste trabalho está centrado na entrega de todas as funcionalidades dos 
aplicativos.


\section{Materiais}	
Foi utilizado um notebook pessoal para o desenvolvimento do projeto. Trata-se 
de um Macbook, pois é necessário para desenvolvimento nativo iOS e atende às demandas para 
desenvolvimento Android.

Os softwares utilizados são gratuítos. Para desenvolvimento iOS, foi utilizado o 
software XCode, e para desenvolvimento Android, o software Android Studio.

\section{Descrição}
O capítulo 2 deste trabalho apresenta alguns conceitos iniciais como interface gráfica 
(UI), experiência do usuário (UX) e API, discutindo algumas tecnologias da camada de 
aplicação da internet em sistemas comumente utilizados por aplicações móveis. 
Também apresenta as etapas de criação de um projeto em ambos os softwares Android 
Studio e XCode.
O capítulo 3, por sua vez, explica mais detalhadamente as interfaces gráficas (ou seja, as "telas")
tanto para Android, quanto para iOS, mostrando suas diferenças e possíveis 
formatos, com respaldo nos conceitos de Activity e de Fragment na plataforma Android e de 
Controller na plataforma iOS. Também é aprofundada a discução sobre
 experiência do usuário, mostrando os conceitos de Intent e FragmentManager para Android e de 
 NavigationController e Segue para iOS.
Já o capítulo 4 descreve a implementação dos aplicativos do RU explicando cada 
componente do projeto bem como as decisões que foram tomadas ao longo de suas implementações.
O último capítulo conclui o trabalho e discute possíveis trabalhos futuros.